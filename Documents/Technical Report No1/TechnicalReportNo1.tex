\documentclass[a4paper, 12pt, twoside]{report}

\usepackage{fontspec}
\usepackage{xunicode}
\usepackage{xltxtra}
\usepackage{xgreek}
\usepackage{csquotes}
\usepackage{placeins}

\raggedbottom
\onecolumn

\usepackage{graphicx, amsfonts, psfrag, fancyhdr, layout, subfigure}
\usepackage{multirow}
\usepackage{longtable}
\usepackage{array}
%The arydshln package offers you the \hdashline and \cdashline commands which are the dashed counterparts of \hline and \cline, respectively. 
\usepackage{arydshln}

\usepackage[toc,page]{appendix}
\renewcommand{\appendixtocname}{Παραρτήματα}
\renewcommand{\appendixname}{Παράρτημα}
\renewcommand{\appendixpagename}{Παραρτήματα}

\setmainfont[Mapping=tex-text]{GFS Artemisia}

\usepackage[style=numeric, bibstyle=numeric, hyperref=true, backref=true, alldates=terse, indexing=false, backend=bibtex]{biblatex}
\addbibresource{Bibliography.bib}

\usepackage{index}
\usepackage[columns=2]{idxlayout}
\newindex{default}{idx}{ind}{Ευρετήριο}

\usepackage{hyperref}

% Set equal margins on book style
\setlength{\oddsidemargin}{53pt}
\setlength{\evensidemargin}{53pt}
\setlength{\marginparwidth}{57pt}
\setlength{\footskip}{30pt}

\author{Καφετζής Δημήτριος Ανδρέας}
\title{Εναέρια λήψη εικόνας - Πρώτη τεχνική έκθεση : Μελέτη σύστασης/κατασκευής πλήρους συστήματος εναέριας λήψης εικόνας}

\begin{document}
	
	\maketitle
	
	\section*{Περίληψη}
		\paragraph{}{Με την αναφορά αυτή επιχειρείται να αποσαφηνιστεί το ίδιο το σύστημα της εναέριας λήψης εικόνας. Παρουσιάζονται τα επί μέρους τμήματα και προτείνονται λύσεις.
		}
		
	\addcontentsline{toc}{chapter}{Περιεχόμενα}
	\tableofcontents

	\newpage
	\phantomsection
	\addcontentsline{toc}{chapter}{Κατάλογος Σχημάτων}
	\listoffigures

	\newpage
	\phantomsection
	\addcontentsline{toc}{chapter}{Κατάλογος Πινάκων}
	\listoftables
	
% Code for creating empty pages
% No headers on empty pages before new chapter
	\makeatletter
		\def\cleardoublepage{
			\clearpage\if@twoside \ifodd\c@page\else
	    	\hbox{}
	    	\thispagestyle{plain}
	    	\newpage
    		\if@twocolumn\hbox{}\newpage\fi\fi\fi
    	}
	\makeatother \clearpage{\pagestyle{plain}\cleardoublepage}

% Define pagestyle
	\pagestyle{fancy}
	\fancyhf{}
	\renewcommand{\chaptermark}[1]{\markboth{ \emph{#1}}{}}

% Adjustments headers
	\fancyhead[LO]{\emph{Κεφάλαιο \thechapter}}
	\fancyhead[RE]{\leftmark}
	\fancyfoot[LE,RO]{\thepage}
	
	\chapter{Εισαγωγή}
		
		\section{Σκοπός}
			\paragraph{}{Σκοπός μας είναι η σύνθεση και κατασκευή ενός συστήματος εναέριας λήψης φωτογραφιών και βίντεο υψηλής ανάλυσης σε ποικίλες καιρικές συνθήκες.
			}
			
		\section{Γενικές απαιτήσεις}
			\paragraph{}{Η κατασκευή και η λειτουργία του συστήματος θα πρέπει να γίνει λαμβάνοντας υπόψιν τους παρακάτω παραμέτρους :
				\begin{itemize}
					\item Ασφάλεια - εξοπλισμού και περιβάλλοντος
					\item Ευχρηστία – πτήσης και λήψης εικόνας
					\item Λειτουργικότητα – πτήσης και λήψης εικόνας
					\item Ποιότητα – εξοπλισμού, πτήσης και λαμβανόμενης εικόνας
					\item Διάρκεια – πτήσης και λαμβανόμενης εικόνας
Κόστος
				\end{itemize}
			}
			
			\paragraph{Ασφάλεια}{Το σύστημα θα πρέπει να είναι ασφαλές, αφενός για τον εαυτό του με σκοπό την προστασία του εξοπλισμού και την αποφυγή καταπόνησης του και αφετέρου για τους χρήστες και το ευρύτερο περιβάλλον στο οποίο θα λειτουργεί.

 			\paragraph{Ευχρηστία}{Το σύστημα όντας εύχρηστο θα μας απαλλάξει από δυσάρεστες καταστάσεις. Απαιτούμενος σκοπός είναι η πτήση και η λήψη εικόνας να γίνονται όσο το δυνατόν ομαλά και ευχάριστα για τους χειριστές. Η εμπλεκόμενες διαδικασίες θα πρέπει να είναι αυτοματοποιημένες σε μεγάλο βαθμό δίνοντας τη δυνατότητα στους χειριστές να επικεντρωθούν στην ποιότητα του τελικού αποτελέσματος.

 			\paragraph{Λειτουργικότητα}{Θα πρέπει να παρέχονται οι κατάλληλες προϋποθέσεις στους χειριστές του συστήματος για την παραγωγή υψηλής ποιότητας εικόνας και μεγάλου αισθητικού ενδιαφέροντος.

 			\paragraph{Ποιότητα}{Ο παράγοντας αυτός αφορά τόσο τα κατασκευαστικά χαρακτηριστικά του εξοπλισμού, όσο και την ποιότητα της πτήσης και της καταγραφόμενης εικόνας. Ο εξοπλισμός οφείλεται να αντέξει στο χρόνο με εμφάνιση ελαχίστων προβληματικών εξαρτημάτων ή υποσυστημάτων και με διακριτική συντήρηση τους. Επιπλέον, η πτήση αναμένεται να είναι τόσο ομαλή και ανεπηρέαστη από τις διάφορες καιρικές συνθήκες (αέρας, βροχή, χιόνι) ώστε να αποτελεί αιτία για λήψη χαμηλής ποιότητας φωτογραφιών και βίντεο. Αναφορικά με την ίδια την ποιότητα της λαμβανόμενης εικόνας επιθυμείτε να είναι η μέγιστη δυνατή αν όχι υψηλής ανάλυσης (high definition).

 			\paragraph{Διάρκεια}{Το σύστημα θα πρέπει να επιτρέπει την αδιάλειπτη καταγραφή εικόνας για μεγάλο χρονικό διάστημα. Ως ενδεικτικός χρόνος συνεχόμενης πτήσης αναφέρονται τα σαράντα (40) λεπτά, ενώ για το χρόνο καταγραφής οι δύο (2)  με δυόμιση (2 και 1/2) ώρες.

 			\paragraph{Κόστος}{Το κόστος απαιτείται να κυμανθεί στο χαμηλότερο δυνατό επίπεδο. Η σύσταση του συστήματος οφείλεται να γίνει λαμβάνοντας υπόψιν τον υπάρχων εξοπλισμό.
 			
 	\chapter{Τεχνική ανάλυση}
		\section{Εισαγωγή}
			\paragraph{}{Στο κεφάλαιο αυτό θα παρουσιάσουμε τα μέρη από τα οποία θα αποτελείται το σύστημα. Θα καταγραφούν τα χαρακτηριστικά που πρέπει να έχει το κάθε τμήμα του ώστε να πληρούνται οι απαιτήσεις που τέθηκαν στο προηγούμενο κεφάλαιο.
			}
			
			\paragraph{}{Ας ξεκαθαρίσουμε από τι μέρη θα περιλαμβάνει το εν λόγω σύστημα.
			\begin{itemize}
				\item πλατφόρμα πτήσης
				\item πλατφόρμα προσγείωσης
				\item αυτόματος πιλότος
				\item χειροκίνητη πτήση
				\item επίγειος σταθμός ελέγχου πτήσης
				\item πλατφόρμα της μηχανής
				\item μηχανή
				\item επίγειος σταθμός ελέγχου της μηχανής
				\item εναέριος σταθμός μετάδοσης εικόνας
				\item επίγειος σταθμός λήψης σύγχρονης εικόνας
				\item επίγειος σταθμός καταγραφής της εικόνας
			\end{itemize}
			}
			
			\paragraph{πλατφόρμα πτήσης}{Αναφέρεται στην πτητική συσκευή. Για λόγους ευστάθειας, ασφάλειας, ευχρηστίας, συντήρησης οδηγούμαστε στην επιλογή ενός ηλεκτρικού μέσου και συγκεκριμένα ενός πολυπτέρου. Το πλήθος των κινητήρων δεν έχει αποσαφηνιστεί. Επιλογές αποτελούν τα εξάπτερα και τα οχτάπτερα.
			}
			\paragraph{πλατφόρμα προσγείωσης}{Αναφέρεται στα "πόδια" του συστήματος. Συνήθως αποτελούν ενιαίο τμήμα με την πλατφόρμα πτήσης.
			}
			\paragraph{αυτόματος πιλότος}{Για την εξασφάλιση τόσο της ασφάλειας πολυπτέρου και περιβάλλοντος, όσο και της ποιότητας του αποτελέσματος κρίνεται απαραίτητος. Διαθέτοντας αυτόν, το πολύπτερο μπορεί άφοβα να χάσει το σήμα και έπειτα από κάποια δεύτερα αν επιστρέψει αυτόματα. Επίσης σημαντικό είναι το γεγονός ότι σε περίπτωση που οι μπαταρίες αδειάσουν, επιστρέφει σε προκαθορισμένο σημείο προσγείωσης με ασφάλεια.
			}
			\paragraph{χειροκίνητη πτήση}{Συνήθως αποτελείται από ένα χειριστήριο και ένα δέκτη τοποθετούμενο στην πλατφόρμα πτήσης. Ο χειριστής θα μπορεί να καθορίζει σε πραγματικό χρόνο την πορεία του πολυπτέρου και να το κατευθύνει κατά την επιθυμία του.
			}
			\paragraph{επίγειος σταθμός ελέγχου πτήσης}{Αποτελεί συσκευή που ενσωματώνει επιπλέον δυνατότητες παρακολούθησης της πτήσης και επέμβασης σε αυτήν. Μπορεί να διαθέτει ειδικό λογισμικό με το οποίο να αποτυπώνεται η πορεία του σε χάρτη (π.χ. google maps) και με το οποίο μπορούν να δοθούν κρίσιμες εντολές όπως άμεση προσεδάφιση, επιστροφή στο σπίτι κ.λ.π.. Επιπλέον μπορεί να διαθέτει λειτουργία OSD -on screen data, δεδομένα στην οθόνη. Δηλαδή, λαμβάνεται βίντεο της πτήσης από την οπτική του πολυπτέρου και επιπλέον απεικονίζονται και διάφορα σημαντικά δεδομένα, όπως υψόμετρο, ταχύτητα ανέμου, ταχύτητα πολυπτέρου κ.λ.π..
			}
			\paragraph{πλατφόρμα της μηχανής}{Στα αγγλικά χρησιμοποιείται ο όρος "gimbal". Αποτελεί το σημείο στήριξης της μηχανής πάνω στην πλατφόρμα πτήσης. Είναι υπεύθυνο για το είδος και μέγεθος της μηχανής, τις επιτρεπτές κινήσεις που μπορεί να εκτελέσει η μηχανή καθώς και για τις δυνατές γωνίες λήψεις. Οι δυνατές κινήσεις είναι η περιστροφή στον κάθετο άξονα -tilt, κίνηση πάνω κάτω- η περιστροφή στον οριζόντιο άξονα -pan, κίνηση δεξιά αριστερά- και η περιστροφή γύρω από τον εαυτό της -roll. Όσο μεγαλύτερο το εύρος των κινήσεων αυτών, τόσο το καλύτερο.
			}
			\paragraph{μηχανή}{Η μηχανή λήψης εικόνας. Για μεγαλύτερη ποιότητα είναι απαραίτητο να τραβάει υψηλής ανάλυσης εικόνα και βίντεο (hd) καθώς και να διαθέτει μεγάλη χωρητικότητα για μεγαλύτερης διάρκεια γύρισμα.
			}
			\paragraph{επίγειος σταθμός ελέγχου της μηχανής}{Αποτελείται από ένα πομπό τοποθετούμενο στο έδαφος και ένα δέκτη τοποθετούμενο στην πλατφόρμα πτήσης σε συνδυασμό με την πλατφόρμα της μηχανής. Ο πομπός συνήθως είναι ένα χειριστήριο που ενσωματώνει τις απαραίτητες λειτουργίες. Ενδεικτικά θα πρέπει να μας επιτρέπει να λαμβάνουμε όποτε επιθυμούμε φωτογραφίες και να καταγράφουμε όποτε επιθυμούμε πάλι βίντεο. Τέλος, θα πρέπει να υποστηρίζει την ενεργοποίηση του φλας.
			}
			\paragraph{εναέριος σταθμός μετάδοσης εικόνας}{Αποτελεί τη συσκευή που μεταδίδει την εικόνα που "βλέπει" η μηχανή με σκοπό να τη λήψη της από επίγειες συσκευές. Θα πρέπει η μετάδοση να είναι ομαλή και απρόσκοπτη και εφικτή για μεγάλες αποστάσεις.
			}
			\paragraph{επίγειος σταθμός λήψης σύγχρονης εικόνας}{Αποτελεί τη συσκευή με την οποία λαμβάνεται η εικόνα που βλέπει η μηχανή στον αέρα και σύμφωνα με την οποία δρα αναλόγως ο χρήστης της μηχανής. Δηλαδή τραβάει ή όχι φωτογραφίες και βίντεο.
			}
			\paragraph{επίγειος σταθμός καταγραφής της εικόνας}{Αποτελεί τη συσκευή η οποία επιτρέπει την καταγραφή εικόνας σε υψηλή ανάλυση (hd). Η μονάδα αποθήκευσης θα πρέπει να εξυπηρετεί τις ανάγκες των γυρισμάτων για μεγάλος πλήθος φωτογραφιών και πολύωρου βίντεο.
			}
			
		\section{Προτεινόμενες λύσεις}
			\paragraph{πλατφόρμα πτήσης}{
			}
			\paragraph{πλατφόρμα προσγείωσης}{
			}
			\paragraph{αυτόματος πιλότος}{
			}
			\paragraph{χειροκίνητη πτήση}{
			}
			\paragraph{επίγειος σταθμός ελέγχου πτήσης}{
			}
			\paragraph{πλατφόρμα της μηχανής}{Σ
			}
			\paragraph{μηχανή}{
			}
			\paragraph{επίγειος σταθμός ελέγχου της μηχανής}{
			}
			\paragraph{εναέριος σταθμός μετάδοσης εικόνας}{
			}
			\paragraph{επίγειος σταθμός λήψης σύγχρονης εικόνας}{
			}
			\paragraph{επίγειος σταθμός καταγραφής της εικόνας}{}
			
			
			
\end{document}